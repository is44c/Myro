\documentclass{letter}[10pt]
\usepackage{url}

\textheight 8.5in

\pagestyle{empty}

\begin{document}

\signature{Institute for Personal Robots in Education}

\address{Institute for Personal Robots in Education\\
Park Science Bldg, Room 251\\
Bryn Mawr College\\
101 North Merion Ave\\
Bryn Mawr, PA 19010\\
Tel. (610) 526-5024}

\begin{letter}{}

\opening{Dear Computer Science Educator,}

The mission of the Institute for Personal Robots in Education,
abbreviated as IPRE, is to find creative, fun, and useful ways for
students to learn computer science by programming robots.  The
institute, created with a grant from Microsoft Research, is jointly
run by Bryn Mawr College and the Georgia Institute of Technology.  For
more information, take a look at IPRE's website:

\begin{small}
\url{http://www.roboteducation.org/}
\end{small}

We have enclosed your robot kit, which includes the following:

\begin{enumerate}
\item Scribbler robot, by Parallax.

\item The `Fluke', IPRE's Bluetooth camera dongle.  Has extra sensors
for the robot, including a camera, and allows the robot to send
messages to and receive messages from the computer.
\item Bluetooth USB dongle.  Allows the computer to send messages to
and receive messages from the robot.
\item A nifty carry bag for the robot.
\item Three Sharpie pens.  For use with Scribbler drawing.
\item Six double-A batteries.  Note that rechargeable batteries work
with the Scribbler as well.
\end{enumerate}

Also included are the original Scribbler manual, cd, and serial cable,
as well as the Bluetooth USB Dongle manual, CD, and USB cable. None of
these last set of items are necessary for programming of the Scribbler
using the Myro library (a set of functions for use with the robot).
We have included them so that in case you wish to use any of them, you
can.

Learning to program the Scribbler in the programming language Python
using the Myro library is fun and relatively easy.  We currently have
introductory level computer science courses at both Bryn Mawr College
and Georgia Technical Institute in which students use the Scribbler to
learn how to program.  For an introduction to the Scribbler and much,
much more information on using Myro with the Scribbler, go to our Wiki
Online Textbook:

\begin{small}
\url{http://wiki.roboteducation.org/Introduction_to_Computer_Science_via_Robots}
\end{small}

To set up the Scribbler for use, you will need to connect the Fluke
into the Scribbler's serial port and connect the Bluetooth USB dongle
into your computer's USB port.  By the serial port there is a black
on-off switch and a useful red restart button.

To use the Scribbler, you will need to download the installer for the
programming language Python, as well as the library Myro, a collection
of functions that can be used with the Scribbler.  Both are freely
available on the web.  For instructions on how to install Myro as well
as on how to set up the Bluetooth connection between your robot and
the computer, go to:

\begin{small}
\url{http://wiki.roboteducation.org/Myro_Installation_Manual}
\end{small}

On that page, there is also a link to Python 2.4, which is what is
used in the Wiki online text.  Alternately, you can get Python 2.4 as
well as newer versions of Python at \url{http://www.python.org/} (go to
Quick Links and look under Releases, both located on the left-hand
side of the webpage).

To make sure that your system is up-to-date, please:

\begin{enumerate}
\item Double-click the ``Start Python.py'' icon on your desktop
\item  Enter ``from myro import *'' in Python
\item Enter ``upgrade()'' and enter your Bluetooth's serial port (ie, COM3).
\end{enumerate}

Occasionally thereafter, you can enter ``upgrade('myro')'' at the Python
prompt to make sure the Python libraries are up-to-date.

Finally, please join our Myro-users mailing list at:

\begin{small}
\url{http://myro.roboteducation.org/mailman/listinfo/myro-users}
\end{small}

We hope you enjoy working, playing, and learning with the Scribbler
robot and Myro!


\closing{Sincerely,}

\end{letter}

\end{document}
